\documentclass[12pt, a4paper]{NGPLB}
\usepackage{tikz}
\usepackage[most, minted]{tcolorbox}
\usepackage{minted}
\usepackage{etoolbox}
\usepackage{indentfirst}
\usepackage{geometry}
\geometry{top = 2cm, bottom = 2cm, left = 2 cm, right = 2cm}
\tcbset{fontupper=\small, breakable, skin=bicolor ,listing engine={minted} ,colback=gray!30!white, colbacklower=gray!5!white, colframe=gray!30!white, minted options ={linenos, breaklines, breakanywhere, showtabs = false, tabsize = 4}}

\newtcblisting{myoutput}{fontupper=\small, breakable, skin=bicolor ,colback=gray!30!white, colbacklower=gray!5!white, colframe=gray!30!white listing engine={minted}, listing only, minted language = python}

\newtcblisting{mylisting}{fontupper=\small, breakable, skin=bicolor ,colback=gray!30!white, colbacklower=gray!5!white, colframe=gray!30!white, listing engine={minted}, listing only, minted language = python}


\linespread{1.35}
\title{Siweb}
\author{Si manglam}
\pagenumbering{arabic}
\setlength{\parindent}{24pt}
\begin{document}
\maketitle
\tableofcontents

\chapter{前言}
我想用 Python 寫一個 Literate Programming 的工具,我現在在用的是 noweb,他有非常簡單的語法,以及強大的 tangle 功能。不過我倒不太喜歡他的 weave 功能,他有寫一個模板,可以照那個模板用,但如果不想按照那個模板就很麻煩。而現在是用 Pweave 來 weave 文件,但 Pweave 只會將程式碼區塊轉成設定的 LaTeX Environment。如果要指定排版還是需要用純 LaTeX 去定義。

我希望寫一個簡單的 Literate Programming 工具,他需要提供以下功能:

\begin{itemize}
\item 如同 noweb 一般簡單的語法
\item 可以自行設定 output Language
\item 對於簡單的 LaTeX 命令可以提供替代
\item 可以提供一個規範,讓撰寫 documentclass 變得方便
\end{itemize}

那由於 Literate Programming 分成兩個部分 Weave \& Tangle 就先讓我們處理比較簡單的 Tangle 吧。

\chapter{Tangle}

Tangle 是負責將程式碼區塊從文件中人類語言的部分剝離出來,我希望 Tangle 可以包含以下功能:

\begin{itemize}
\item 可以將 Root Chunk 內的 Chunk 也一併剝離出
\item 可以同時剝離出多個 File
\end{itemize}

大概就是這兩個功能最重要了,那麼接下來就要來實作這些功能了,那 sitangle 的檔案結構如下:


\begin{mylisting}
#!/usr/bin/python3
<<module>>
<<全域變數>>
<<處理參數>>
<<讀取檔案中並儲存程式碼區塊>>
<<輸出檔案>>
\end{mylisting}


\section{讀取檔案中並儲存程式碼區塊}

首先,我們先來處理讀取檔案中並儲存程式碼區塊的功能,這部分需要先將程式碼區塊的開始與結束定義出來,目前的想法是套用 noweb 的語法,所以讓我們先把判定規則給寫好再去想後面的事情。


\begin{mylisting}
with open(file_name, "r") as f:
	for line in f:
		<<判斷程式碼區塊>>
		<<符合>>
		<<不符合>>
\end{mylisting}


先打開檔案後,慢慢的一行一行讀取檔案中的資料,如果檔案中有符合 <<(程式碼區塊名字,不可含 > )>>= 就會被接下來的判斷式帶到符合或不符合作相應的處理。那接下來先講符合之後該怎麼處理。因為我們想要可以一次就 tangle 出所有所有檔案的功能,所以我們勢必要設定一個全域變數來追蹤 Root Chunk 並在最後對所有 Root Chunk 都進行處理,另外就是我們還要將每一個 Chunk 都儲存起來,並在需要的時候讀取。

我們會先講講該如何儲存每一個 Chunk 再說說該如何根據不同的情況做處理。我們會需要追蹤以下三件事情,目前在的 Chunk、所有的 Chunk 與他們的內容還有所有的 Root\_Chunk,設三個全域變數去追蹤這些事情:


\begin{mylisting}
root_chunk = []
chunks = {}
current_chunk = None
\end{mylisting}


root\_chunk 是負責追蹤所有的 Root Chunks 在需要時可以當作 tangle 檔案的參數,使用 List 來儲存,讓之後可以用 for 迴圈去 Loop。chunks 是用來處理所有的 Chunks 內的內容,因為需要一個一對一的映射關係,所以採用 Dictionary 來儲存他們的名字與值。最後是 current\_chunk 這是負責記錄當前所在的 Chunk,在非 None 的情況下,就代表目前讀到了 Chunk 需要將目前讀到的東西送進這個 Chunk 中,由於 Python 讀檔案是讀成 List 所以我們也用 List 去儲存 Chunk 內的東西。

接下來是判斷是否掃到程式碼區塊,並根據結果進行不同的處理。首先是判斷,這部分會用到 Regex 來協助判斷,所以需要先使用 Python 的 re module 以提供 Regex 的支持:


\begin{mylisting}
import re
\end{mylisting}


接下來就可以撰寫判斷式了,由於我們要參考的是 noweb 的簡單語法,其語法是 <<(程式碼區塊名稱,不可含 > )>>=,寫成 Regex 就會變成 <<([\^>])>> 這樣,限制是名字內不可以有 >,但可以確保不會不小心涵蓋到過多的範圍。中間用括號包起來的原因是可以把裡面的內容包成一個 group,並可以利用 group 把 Chunk 名稱記下來:


\begin{mylisting}
match = re.match(r"<<([^>]+)>>=", line)
\end{mylisting}


如果判斷到程式碼區塊就會先查看是否是 Root Chunk,若是便會被加進 root\_chunk 中以便追蹤所有的 Root Chunk,接下來會先判斷現在的 chunk 是否已被讀取過,若沒有便會先在 Chunks 中創建一個與程式碼區塊名稱對應的 List,不然會 KeyError。然後就會將 current\_chunk 設成目前讀到的程式碼區塊名稱,讓接下來的東西可以被送進對的區塊內:


\begin{mylisting}
if match:
	current_chunk = match.group(1)
	if not current_chunk in chunks:
		chunks[current_chunk] = []
	if re.match(r"\S*\.\S+", current_chunk):
		if not current_chunk in root_chunk:
			root_chunk.append(current_chunk)
\end{mylisting}


接下來就要來處理不符合的情況,不符合會有兩種情況,第一是已經有掃描到 Chunk 要把接下來的內容送到對應的 Chunk 內,第二是沒有掃到 Chunk 就是用來描述的人類語言。第一種會需要把它存到 current\_chunk 內,第二種就不需要特殊處理。


\begin{mylisting}
else:
	if current_chunk:
		match = re.match(r"@[^\S]", line)
		if match:
			current_chunk = None
		else:
			chunks[current_chunk].append(line)
\end{mylisting}


判定的規則是如果已經讀到了 Chunk 就判定 Chunk 是不是要結束了,如果有單獨的 @ 被讀取到就會把 current\_chunk 設為 None 並跳出,如果沒有就會把目前讀到的東西丟進 chunks 中儲存。

\section{輸出檔案}

輸出檔案會分成兩個部分,第一個是遞迴函數負責將 Chunk 裡面包含的子 Chunk 給剝離出來,另外一部分則是將剝離出來的東西給寫入檔案。除此之外,我們還需解析輸入的參數,若輸入的參數為 all 便需要 tangle 全部的 Root Chunks。所以先讓我們從處理參數開始,再慢慢到輸出檔案的部分。

處理參數使用了 Python 的 sys module,我們必須先 import 這個 module 再進行進一步的分析。另外,我們也需要處理不按照規則的使用者輸入,所以也額外導入 os 來幫助我們直接結束程式。


\begin{mylisting}
import os, sys
\end{mylisting}


我們的 input 希望可以越簡單越好,目前的想法是想以下這樣 \verb|sitangle chunkname file| 簡簡單單即可。而另設有一保留字 all 只要使用這個就會自動把所有的檔案產生出來。使用 sys.agrv 可以獲取 command line 中指定的參數,並會以 List 的方式供開發者使用。List 的第一項會是檔案的絕對路徑,第二項後才是輸入的文字,為求簡單我們這裡先不考慮以 \-\- 開始的可選選項,先對非 \-\- 開頭的事物做判定。


\begin{mylisting}
for i in sys.argv[1:]:
	if i.startswith("--"):
		<<處理可選選項>>
	else:
		if output_chunk and file_name:
			print("Error, More than two argument")
			exit(1)
		elif output_chunk:
			file_name = i
		else:
			output_chunk = i
\end{mylisting}


這裡設置了兩個全域變數,分別是 file\_name 與 output\_chunk 這兩個變數。這兩個變數的預設皆為 None,讀到的第一個非可選選項會先被指定給 output\_chunk, 如果兩個都被設置了卻還有多的參數,就會跳出錯誤訊息並關閉程式。


\begin{mylisting}
file_name = None
output_chunk = None
\end{mylisting}


\subsection{輸出檔案}

首先,先讓我們把檔案輸出的部分寫好,我們需要把程式碼剝離出來,並需特別注意兩項事情,第一件是子區塊,需要一個遞迴函式去將其剝離出來,第二是程式碼的縮排,因為有些程式語言對縮排是有特殊要求的。所以先讓我們寫一個用來剝離程式碼的遞迴函數,並同樣利用 Regex 來判定是否有子區塊,若有便 Return 自身,若無便將此行加入 result 中,順便利用例外處理去處理區塊未定義的問題。


\begin{mylisting}
def expand(chunk_name, indent):
	result = []
	try:
		output_chunk = chunks[chunk_name]
	except KeyError:
		print(f"KeyError {chunk_name} undefined")
		return []
	for i in output_chunk:
		match = re.match(r"(\s*)<<([^>]+)>>\s*$", i)
		if match:
			result.extend(expand(match.group(2), indent + match.group(1)))
		else:
			result.append(indent + i)
	return result
\end{mylisting}


那接下來就是把剝離出的程式碼輸入進檔案:


\begin{mylisting}
if output_chunk == "all":
	for name in root_chunk:
		os.system(f"touch {name}")
		with open(name, "w+") as f:
			for line in expand(name, ""):
				f.write(line)
else:
	with open(output_chunk, "w+") as f:
		for line in expand(output_chunk, ""):
			f.write(line)
\end{mylisting}


最後是處理可選選項,其實也沒有太多可選可選,僅有一個 help 可選。不實作過多的選項是因為過於麻煩,且希望這是一個簡單的工具。


\begin{mylisting}
i = i.strip("--")
if i == "help":
	print("This is a Literate Programming tool made by Si Manglam to satisfy his own requirement.\nUsage: sitangle output_chunk_name file")
	exit(0)
else:
	print("Unknown Option")
	exit(1)
\end{mylisting}


\chapter{weaver}

接下來進入到 weave 的實作,由於最近有幸去看 Donald E. Knuth 的 ``Literate Programming'',發現他的排版確實好看,且有標示出檔案的父區塊與其他區塊的所在位置,非常的有幫助。如果想要實現這個功能,我們會需要一種可以儲存上下級資訊的資料結構,還需要另外一資料結構來儲存同名區塊的順序與標籤。我想這兩個應該可以用樹去儲存資訊,樹負責儲存上下級關係,順序則直接記得出現次數,並把標籤編上號碼即可。

所以我們需要先把樹跟鏈結串列給刻出來,再來去思考後面的東西。那整個檔案結構會看來如下:


\begin{mylisting}
<<節點>>
<<樹>>
<<全域變數>>
<<Debug>>
\end{mylisting}


想法是先掃瞄過整個檔案後,建立好所有的節點,每個節點都包含了以下資訊:父節點、子節點與出現次數。在將所有節點都建立好後才會把整個樹串起來,串起來的方式是先找到根節點再慢慢去找其子節點的名稱,並利用子節點的名稱將節點串起來。串起來之後才會進行輸出,但這樣需要至少做兩次檔案讀寫,之後說不定可以找到更好的方法

\section{樹}

首先我們需要先刻一顆樹出來,我需要兩個資訊,分別是父節點的名稱與其有的子節點。樹還需要提供 Merge 與 Search 的功能,在掃描過整個

\end{document}
